\documentclass{emulateapj}
\bibliographystyle{astroads}

%define general packages
\usepackage{epsfig}
\usepackage{amsmath}
\usepackage{natbib}

\usepackage{xcolor}
\newcommand\writingnote[1]{\textcolor{red}{#1}}
% \renewcommand\writingnote[1]{}

\begin{document}

\title{Isocurvature Forecasts for Planck, CMB S4, and PIXIE, and Maybe Constraints for ACTPol}
\author{Zack Li and Jo Dunkley}
\date{\today}
%\maketitle
\affil{Astrophysical Sciences, Princeton University, Princeton NJ 08544}


\begin{abstract}
We provide forecasts of cold dark matter isocurvature (CDI) constraints for combinations of Planck, CMB S4, and PIXIE. Using MCMC methods on fiducial power spectra, we find substantial improvements in the measurement of the large scale isocurvature power. 
\end{abstract}

%Section heading
\section{Introduction}

The primordial cosmological perturbations are primarily adiabatic fluctuations, which come from a spatially uniform equation of state and initial velocity field and lock together the density perturbations of the different components \citep{planckXX:2013}.  Several scenarios allow for spatially varying equations of state or initial velocity fields, producing isocurvature perturbations. 

\writingnote{Motivations.}
Inflation with a single scalar field and slow-roll initial conditions excites only adiabatic perturbations. However, multiple field inflation can produce an isocurvature spectrum as well as an adiabatic spectrum, with possible correlations between the two \citep{langlois:1999}. Commonly studied isocurvature perturbations arise from variations between photon density, cold dark matter (CDM) density, neutrino density (NID), and neutrino velocity (NIV). Quantum fluctuations can also lead to the curvaton scenario, which also generating isocurvature perturbations correlated with the adiabatic modes \citep{baumann/etal:2009}. Some string theory axions can also carry isocurvature fluctuations from quantum fluctuations, with an axion decay to dark matter leading to uncorrelated adiabatic and CDM isocurvature perturbations.

In this paper we forecast constraints on CDM isocurvature using MCMC on fiducial power spectra and simulations of future CMB experiments. The CDM isocurvature contribution to TT, TE, and EE power spectra is out of phase with the adiabatic perturbations in $C_l$, so improvements in CMB polarization measurements can considerably improve upon current constraints (see Section \writingnote{X.X}). We simulate combinations of Planck, CMB S4, and PIXIE. \writingnote{Should I cite each experiment here?}


\writingnote{Current Constraints.}
Current measurements of isocurvature are consistent with fully adiabatic primordial density fluctuations. Previous constraints on isocurvature constraints have come from precision measurements of the CMB power spectrum, with WMAP \citep{moodley/etal:2004} and Planck \citep{planckXX:2013}. Joint WMAP and BAO constraints on fully uncorrelated and anticorrelated modes put upper bounds on the isocurvature at less than a percent \citep{hinshaw/etal:2013}. \writingnote{Should I include actual numbers?} The tightest constraints on isocurvature \citep{planckXX:2015} currently limit the amplitude of CDM isocurvature fluctuations to a few percent at large scales ($k=0.002$ Mpc$^{-1}$), and 30-50\% at small scales ($k=0.1$ Mpc$^{-1}$). These use polarization measurements over the Planck range of $l=2-2508$ for TT and $2-1996$ for TE and EE. 


\writingnote{Recent Forecasting Papers.} 
\cite{bucher/etal:2001} forecasted isocurvature constraints for Planck, the current best constraints on isocurvature using a Fisher matrix analysis. These forecasts have been shown to be consistent with actual measurements. Simulated forecasts for only CMB-S4 \citep{CMB-S4:2016} predict increases in sensitivity by a factor of 3-6 over Planck. For the curvaton scenario, \cite{smith/grin:2016} make forecasts using an MCMC analysis for a cosmic-variance limited experiment. \cite{kasanda/moodley:2014} make isocurvature forecasts for the PRISM experiment and investigate the effects of the CMB lensing potential in-depth.

\writingnote{
todo!
\begin{itemize}
\item \textbf{Why did we only work with CDM isocurvature?}
\item \textbf{Why didn't Planck (or we) add in BAO likelihoods?}
\end{itemize}
}

\section{Methods}


\subsection{Perturbations and Power Spectra}

\writingnote{
todo: heuristic description of isocurvature effects on CMB power spectra. maybe I include something like the $dD_l/dP_{II}^j$ plots.
}

\writingnote{
I should include the description of PII and PRI.
}


For a set of the standard cosmological parameters with the additional isocurvature parameters, we compute a theoretical power spectrum with CLASS, a fast Boltzmann code written in C (citation). The adiabatic and isocurvature are contained in three functions, $\mathcal{P}_{\mathcal{RR}}(k)$, $\mathcal{P}_{\mathcal{II}}(k)$, and $\mathcal{P}_{\mathcal{RI}}(k)$, the curvature, isocurvature, and cross-correlation power spectra, respectively (cite Planck 2015 XX). Like Planck, we specify these power spectra through two scales, $k_1 = 0.002$ Mpc$^{-1}$ and $k_2 = 0.100$ Mpc$^{-1}$. We use the same uniform priors as Planck,
\begin{equation}
    \mathcal{P}_{\mathcal{RR}}^{(1)}, \mathcal{P}_{\mathcal{RR}}^{(2)} \in (10^{-9}, 10^{-8}),
\end{equation}
\begin{equation}
    \mathcal{P}_{\mathcal{II}}^{(1)}, \mathcal{P}_{\mathcal{II}}^{(2)} \in (0, 10^{-8}),
\end{equation}
\begin{equation}
    \mathcal{P}_{\mathcal{RI}}^{(1)} \in (-10^{-8}, 10^{-8}).
\end{equation}
We follow Planck 2015 XX's convention of fixing $\mathcal{P}_{\mathcal{RI}}^{(2)}$ from these parameters. Then we sample over the $\Lambda$CDM scenario, but replace $A_s$ and $n_s$ with $\mathcal{P}_{\mathcal{RR}}^{(1)}$,  $\mathcal{P}_{\mathcal{RR}}^{(2)}$ and add the three isocurvature parameters $\mathcal{P}_{\mathcal{II}}^{(1)}$, $\mathcal{P}_{\mathcal{II}}^{(2)}$, $\mathcal{P}_{\mathcal{RI}}^{(1)}$.
\begin{align}
\{ \Omega_b h^2, \Omega_c h^2, \theta_A, &\tau_{reio}, \mathcal{P}_{\mathcal{RR}}^{(1)}, \mathcal{P}_{\mathcal{RR}}^{(2)} \\
& \mathcal{P}_{\mathcal{II}}^{(1)}, \mathcal{P}_{\mathcal{II}}^{(2)}, \mathcal{P}_{\mathcal{RI}}^{(1)}    \}
\end{align}

\writingnote{Should I describe the perturbation stuff of how the CLASS isocurvature code works?}

\subsection{Forecasting}
\begin{deluxetable*}{cccccc}
\tabletypesize{\footnotesize} 
\tablecolumns{6} 
\tablewidth{0pt} 
\tablecaption{ Forecasting Parameters \label{table:forecasts}} 
\tablehead{ 
 \colhead{Experiment}     & \colhead{$l_{min}$ - $l_{max}$}  & \colhead{$f_{sky}$} & \colhead{$\theta$ FWHM} & \colhead{$\sigma_T$ ($\mu$K arcmin)} &  \colhead{$\sigma_P$ ($\mu$K arcmin)} }
 \startdata 
  CMB S4          & 30-3000 & 0.40 & 3.0 & 1.0 & 1.4\\
  PIXIE           & 2 - 150 & 0.8 & 120 & 2.9 & 4.0 \\
  Planck 2017 high\_l & 30 - 2500 & 0.65 & 10,7.1,5.0 & 65.0, 43.0, 66.0 & 103.0, 81.0, 134.0 \\
\enddata 
% \vspace{-0.8cm} 
\tablecomments{These forecasts are based on \writingnote{blahblahblah}.} 
\end{deluxetable*}

\writingnote{
    \begin{itemize}
        \item explanation of how we got these numbers in Table \ref{table:forecasts}
        \item how do you turn the numbers in Table \ref{table:forecasts} into a likelihood
        \item atmospheric noise model
        \item choosing a fiducial power spectrum. 
        \item comparing the fiducial power spectrum Planck parameter estimates with real Planck (do we we need this?)
    \end{itemize}
}




\subsection{ACTPol Likelihood}

We use the same methods as in Louis et al. 2016 for the ACT likelihood, marginalizing the ACTPol spectrum from $350 < l < 4000$ to construct a Gaussian likelihood function with an overall calibration parameter. We produce our parameter constraints by summing this with the Planck 2015 log-likelihood. We use the public CMB-marginalized 'plik-lite' Planck 2015 likelihood which uses TT for $30 \leq l \leq 2508$, a likelihood generated from CMB lensing, and a joint TT, EE, BB, and TE likelihood for the range $2 \leq l < 30$. 

\begin{align}
-2 \ln L = &- 2 \ln L(\text{ACTPol}) \\
&-2 \ln L(\text{Planck TT}_{30 < l < 2508}) \\ 
&-2 \ln L(\text{Planck TEB}_{2 \leq l < 30})\\ 
&-2 \ln L(\text{Planck Lensing})
\end{align}

In addition to the $\Lambda$CDM and isocurvature parameters, we need two nuisance parameters coming from the normalizations of the two instruments we use data from (Planck and ACT),
\begin{equation}
\{ A_{planck}, Y_p \}.
\end{equation}

\section{Results}

\writingnote{
    \begin{itemize}
        \item a triangle plot with all of the forecasts
        \item derived parameters plot, overplot all forecasts
        \item ACTPol measurements 
    \end{itemize}
}

\section{Conclusion}


\bibliography{mybib}

\end{document}


