\documentclass[11pt,a4paper]{emulateapj}
\bibliographystyle{apj}

%define general packages
\usepackage{epsfig}
\usepackage{amsmath}
\usepackage{natbib}

\usepackage{xcolor}
\newcommand\myworries[1]{\textcolor{red}{#1}}
% \renewcommand\myworries[1]{}

\begin{document}

\title{Isocurvature Forecasts for CMB S4, PIXIE, and Maybe Constraints for ACTPol}
\author{Zack Li and Jo Dunkley}
\date{\today}
%\maketitle


\begin{abstract}
This is a sample document which demonstrates some of the basic features
of \LaTeX.  You can easily reformat it for different document or bibliography styles.
\end{abstract}

%Section heading
\section{Introduction}

\begin{itemize}
\item 
\end{itemize}

\section{Methods}

\subsection{Perturbations and Power Spectra}
For a set of the standard cosmological parameters with the additional isocurvature parameters, we compute a theoretical power spectrum with CLASS, a fast Boltzmann code written in C (citation). The adiabatic and isocurvature are contained in three functions, $\mathcal{P}_{\mathcal{RR}}(k)$, $\mathcal{P}_{\mathcal{II}}(k)$, and $\mathcal{P}_{\mathcal{RI}}(k)$, the curvature, isocurvature, and cross-correlation power spectra, respectively (cite Planck 2015 XX). Like Planck, we specify these power spectra through two scales, $k_1 = 0.002$ Mpc$^{-1}$ and $k_2 = 0.100$ Mpc$^{-1}$. We use the same uniform priors as Planck,
\begin{equation}
    \mathcal{P}_{\mathcal{RR}}^{(1)}, \mathcal{P}_{\mathcal{RR}}^{(2)} \in (10^{-9}, 10^{-8}),
\end{equation}
\begin{equation}
    \mathcal{P}_{\mathcal{II}}^{(1)}, \mathcal{P}_{\mathcal{II}}^{(2)} \in (0, 10^{-8}),
\end{equation}
\begin{equation}
    \mathcal{P}_{\mathcal{RI}}^{(1)} \in (-10^{-8}, 10^{-8}).
\end{equation}
We follow Planck 2015 XX's convention of fixing $\mathcal{P}_{\mathcal{RI}}^{(2)}$ from these parameters. Then we sample over the $\Lambda$CDM scenario, but replace $A_s$ and $n_s$ with $\mathcal{P}_{\mathcal{RR}}^{(1)}$,  $\mathcal{P}_{\mathcal{RR}}^{(2)}$ and add the three isocurvature parameters $\mathcal{P}_{\mathcal{II}}^{(1)}$, $\mathcal{P}_{\mathcal{II}}^{(2)}$, $\mathcal{P}_{\mathcal{RI}}^{(1)}$.
\begin{align}
\{ \Omega_b h^2, \Omega_c h^2, \theta_A, &\tau_{reio}, \mathcal{P}_{\mathcal{RR}}^{(1)}, \mathcal{P}_{\mathcal{RR}}^{(2)} \\
& \mathcal{P}_{\mathcal{II}}^{(1)}, \mathcal{P}_{\mathcal{II}}^{(2)}, \mathcal{P}_{\mathcal{RI}}^{(1)}    \}
\end{align}

\myworries{Should I describe the perturbation stuff of how the CLASS isocurvature code works?}



\subsection{Forecasting}


\begin{deluxetable*}{cccccc}[h]
\tabletypesize{\footnotesize} 
\tablecolumns{6} 
\tablewidth{0pt} 
\tablecaption{ Forecasting Parameters \label{table:forecasts}} 
\tablehead{ 
 \colhead{Experiment}     & \colhead{$l_{min}$ - $l_{max}$}  & \colhead{$f_{sky}$} & \colhead{$\theta$ FWHM} & \colhead{$\sigma_T$ ($\mu$K arcmin)} &  \colhead{$\sigma_P$ ($\mu$K arcmin)} }
 \startdata 
  CMB S4          & 30-3000 & 0.40 & 3.0 & 1.0 & 1.4\\
  PIXIE           & 2 - 150 & 0.8 & 120 & 2.9 & 4.0 \\
  Planck 2017 high\_l & 30 - 2500 & 0.65 & 10,7.1,5.0 & 65.0, 43.0, 66.0 & 103.0, 81.0, 134.0 \\
\enddata 
% \vspace{-0.8cm} 
\tablecomments{These forecasts are based on \myworries{blahblahblah}.} 
\end{deluxetable*}



\subsection{ACTPol Likelihood}

We use the same methods as in Louis et al. 2016 for the ACT likelihood, marginalizing the ACTPol spectrum from $350 < l < 4000$ to construct a Gaussian likelihood function with an overall calibration parameter. We produce our parameter constraints by summing this with the Planck 2015 log-likelihood. We use the public CMB-marginalized 'plik-lite' Planck 2015 likelihood which uses TT for $30 \leq l \leq 2508$, a likelihood generated from CMB lensing, and a joint TT, EE, BB, and TE likelihood for the range $2 \leq l < 30$. 

\begin{align}
-2 \ln L = &- 2 \ln L(\text{ACTPol}) \\
&-2 \ln L(\text{Planck TT}_{30 < l < 2508}) \\ 
&-2 \ln L(\text{Planck TEB}_{2 \leq l < 30})\\ 
&-2 \ln L(\text{Planck Lensing})
\end{align}

In addition to the $\Lambda$CDM and isocurvature parameters, we need two nuisance parameters coming from the normalizations of the two instruments we use data from (Planck and ACT),
\begin{equation}
\{ A_{planck}, Y_p \}.
\end{equation}

\section{Results}

\bibliography{example}

\end{document}
